%%%%%%%%%%%%%%%%%%%%%%%%%%%%%%%%%%%%%%%%%%%%%%%%%%%%%%%%%%%%%%%%%%%%%%%%%%%%%
%%
%% This file provides a template that can be used in concert with the
%% ohio-etd class to generate an electronic thesis or dissertation which
%% meets the formatting requirements at Ohio University.
%%
%% To use the template, copy this file (template.tex) and ohio-etd.cls into
%% the same directory and edit this template as required.  Reference
%% ohio-etd.pdf for additional instructions on using this class.
%%
%%%%%%%%%%%%%%%%%%%%%%%%%%%%%%%%%%%%%%%%%%%%%%%%%%%%%%%%%%%%%%%%%%%%%%%%%%%%%


%% Load the class.  Available options are: numbered, pdftex, cmfont,
%% singlespacetables, draft, 11pt, 12pt, leqno, and fleqn

\documentclass[numbered,pdftex]{ohio-etd}
\usepackage{cite}
\usepackage{notoccite}
\usepackage{amsmath} % Inserting Equations
\usepackage{ragged2e}
\usepackage{verbatim} 
\usepackage{hyperref}
\usepackage{flafter}
\usepackage{color,soul} % Highlighting 
\usepackage{subfigmat}

\usepackage{float} % In-text figs
\usepackage{varioref}%  smart page, figure, table, and equation ref
\usepackage{graphicx} % Include graphics
\usepackage{epstopdf} % Figure type .eps to .pdf
\usepackage{wrapfig} % Figures in text
\usepackage{listings}
\usepackage{color} %red, green, blue, yellow, cyan, magenta, black, white
\definecolor{mygreen}{RGB}{28,172,0} % color values Red, Green, Blue
\definecolor{mylilas}{RGB}{170,55,241}
%% Other packages that may be of use.  Delete or comment out (using a
%% percent sign in the first column) if they are not desired.  Reference
%% the corresponding documentation for more information on how to use these
%% packages.

%\usepackage[square,sort&compress,numbers]{natbib} % Provides formatting for
                                                  % citations
\usepackage{textcomp} % Provides math symbols that can be used in text mode
\usepackage{amssymb}  % Provides additional AMS math symbols.  Note that
                      % amsmath is loaded as part of the ohio-etd class
\usepackage{bm}       % Provides bold-faced math symbols
\usepackage{booktabs} % Provides improved table formatting
\usepackage{dcolumn}  % Provides table columns aligned at decimal points
\usepackage{multirow} % Provides table elements spanning multiple rows
\usepackage{graphicx} % Standard package to incorporate graphics
\usepackage[printonlyused]{acronym} % Provides a method for incorporating
                                    % acronyms and building an acronym list

\graphicspath{{figures/}} % Allows graphics files to be stored in a
                          % separate directory
\usepackage{fancyref}
\usepackage{}

%% Required front matter definitions

\degree    {MS}              % MS, MA, MCTP, or PhD
\graduation{December}{2017}    % May, August, or December 

\title     {TITLE}
\author    {Garrett S.}{Clem} 

\advisor   {Jay P. Wilhelm}{Assistant Professor of Mechanical Engineering}
\dean      {Dennis Irwin}{Dean, Russ College of Engineering and Technology }
\program   {Mechanical Engineering}            % e.g. Electrical Engineering
\department{Department of Mechanical Engineering} % e.g. School of Electrical Engineering 
                                    %      and Computer Science
\college   {Russ College of Engineering and Technology}   % e.g. Russ College of Engineering and
                                     %      Technology
\abstract  {ABSTRACT}


%% Optional front matter definitions.  Delete or comment out if not needed

% \coadvisor {Coadvisor's Full Name}{Coadvisor's Full Title}

\dedication{DED}

\acknowledgments{ACK}
%% If you prefer to provide "acknowledgements" instead (note the added "e"
%% between the "g" and the "m") then add the "e" in the macro name so that
%% it reads "\acknowledgements".}

%% Additional "lists" can be added to the end of the front matter using the
%% \addlistof macro.  For example:
\addlistof{Symbols}{\begin{tabbing}
  XXXX \= \kill% this line sets tab stop
  \\
  $a$ \> Previous or Initial Axial Induction Factor (-) \\
  

 \end{tabbing}}
%% Note that the command "\input{symbols}" can be used if the symbol list is
%% contained in a separate file called "symbols.tex"}

\addlistof{Acronyms}{

\begin{tabular}{lll}
AoA & Angle of Attack & \\
\end{tabular}}


%% Use "\input{acronyms}" if the acronym list is in a separate file called
%% "acronyms.tex".  Note that the formatting generated by the acronym package
%% can be forced into singlespaced text by inserting "\setlength\itemsep{0pt}
%% \setlength\parskip{0pt}" into the "acronym" environment.} 

%% For documents created by government employees as part of their
%% employment.  The wording of the disclaimer can be specified using an
%% option.  See the documentation for more information.

% \govtdisclaimer    

% \notables  % Prevent a list of tables from being created
% \nofigures % Prevent a list of figures from being created

\begin{document}

\makefrontmatter    % Creates all of the front matter pages.

% for matlab code entrys
\lstset{language=Matlab,%
    %basicstyle=\color{red},
    breaklines=true,%
    morekeywords={matlab2tikz},
    keywordstyle=\color{blue},%
    morekeywords=[2]{1}, keywordstyle=[2]{\color{black}},
    identifierstyle=\color{black},%
    stringstyle=\color{mylilas},
    commentstyle=\color{mygreen},%
    showstringspaces=false,%without this there will be a symbol in the places where there is a space
%     numbers=left,%
%     numberstyle={\tiny \color{black}},% size of the numbers
%     numbersep=6pt, % this defines how far the numbers are from the text
    emph=[1]{for,end,break},emphstyle=[1]\color{red}, %some words to emphasise
    %emph=[2]{word1,word2}, emphstyle=[2]{style},    
}

%% Body of the text follows, using \chapter, \section, \subsection,
%% \subsubsection, \paragraph, and \subparagraph to generate the
%% section headings.  For convenience, it may be useful to break the
%% full document into separate files, perhaps divided by chapters.  In
%% that case, the files would be loaded here using "\input{filename}"



%%UAVs commercially available, ease of use, inexpensive, open source community, wells supported
%- Unmanned Aerial Vehicles (UAVs) are a powerful robotics tool for military and civilian communities alike
%- Advances in lightweight materials and commercial electronics have increased the payload, range, and overall capability of the crafts
%- Reduced costs have brought the technology to a wide range of communities which have found uses such as surveillance, reconnaissance, aerial photography, delivery, and for defense.
%- As the complexity of tasks increase, the complexity of guidance and control systems increase as well. 
%- Traditionally, guidance and control laws have been separate systems responsible for independent tasks
%- Potential field and vector field methods have blurred the line between the guidance and control systems where they have shared the burden of more complex tasks such as obstacle avoidance and path following
%- Potential field has been effective for obstacle avoidance and goal seeking
%- Fixed wing UAVs have a stall velocity constraint and can not stop at a given point
%- Vector field methods have been effective at providing guidance and control for fixed wing UAVs for converging to and following a path
%- Vector fields can be generated in a number of ways, however a convenient method first introduced in \cite{goncalves_artificial_2009} constructs the field by calculating the integral lines converging to and following the level sets of intersecting surfaces. The intersection of surfaces gauntnesses convergence and provides a scalar multiplication factor for convergence, circulation, and time-varying components of the field. 

\chapter{Introduction}
\section{Motivation and Problem Statement}

%%UAVs commercially available, ease of use, inexpensive, open source community, wells supported
%- Unmanned Aerial Vehicles (UAVs) are a powerful robotics tool for military and civilian communities alike
%- Advances in lightweight materials and commercial electronics have increased the payload, range, and overall capability of the crafts
%- Reduced costs have brought the technology to a wide range of communities which have found uses such as surveillance, reconnaissance, aerial photography, delivery, and for defense.
%- As the complexity of tasks increase, the complexity of guidance and control systems increase as well. 
%- Traditionally, guidance and control laws have been separate systems responsible for independent tasks
%- Potential field and vector field methods have blurred the line between the guidance and control systems where they have shared the burden of more complex tasks such as obstacle avoidance and path following
%- Potential field has been effective for obstacle avoidance and goal seeking
%- Fixed wing UAVs have a stall velocity constraint and can not stop at a given point
%- Vector field methods have been effective at providing guidance and control for fixed wing UAVs for converging to and following a path
%- Vector fields can be generated in a number of ways, however a convenient method first introduced in \cite{goncalves_artificial_2009} constructs the field by calculating the integral lines converging to and following the level sets of intersecting surfaces. The intersection of surfaces gauntnesses convergence and provides a scalar multiplication factor for convergence, circulation, and time-varying components of the field. 

%Unmanned Aerial Vehicles (UAVs) are powerful robotics tools used by both military and civilian communities alike. Advances in lightweight materials and commercial electronics have increase the range, payload, and reliability of the crafts. Reduced costs have brought the technology to a wide range of communities which have found uses such as surveillance, reconnaissance, and aerial photography to name a few. As the complexity of the tasks increase, the complexity of the guidance and control systems increase as well. Traditionally, guidance and control laws have been regarded as separate systems responsible for independent tasks. Potential field and vector field methods have blurred the line between guidance and control systems, whereas now they have shared the burden of complex tasks such as obstacle avoidance and path following. Potential field has been effective for obstacle avoidance and goal seeking for a singular discrete point, however is not initially intended to be used for tasks such as path following. Vector field methods have been effective at providing guidance and control for fixed wing UAVs for converging to and following a path. Vector fields can be constructed in a number of ways, however a convenient method first introduced in \cite{goncalves_artificial_2009} constructs a field by summing together three components consisting of convergence, circulation, and time varying. The filed is generated by calculating the integral lines converging to and following the level sets of intersecting arbitrary surfaces in n-dimensions. Unlike potential field, vector fields do not take into account the dynamics of a vehicle being provided the guidance which provides no guarantee that the UAV will avoid obstacles. Improving the guidance provided by the vector field for obstacle avoidance may be possible by modifying the circulation term of a vector field as a function of a vehicles state with respect to the obstacle.


% Increased availability, used in many communities 
% Usefullness of UAVs has yet to see its limit
% Tasks are becoming more complicated
% Turning high level tasks into lower level guidance and control efforts . . .
% Traditional controls complex
% What is a potential field
% What is a vector field
% Vector field has found a lot of use in converging to and following a path


% Field can be used to repel vehicles and placed in a global vector field to represent an obstacle or no-fly zone
% Summing the vector fields into a resultant field gives the UAV guidance to goal while avoiding obstacle
% Combindation of vector fields can lead to singularities
% Additionally, for a UAV with minimum forward velocity and minimum turning radius, there is no guaruntee the guidance will prevent the UAV from entering the no-fly zone
% Gonsalves Circulation VF
% The proposed research seeks to improve the use of vector field obstacle avoidance by eliminating singularities and saturated control conditions through the use of a scalar time paramatrized circulation function. 

%Unmanned Aerial Vehicles (UAVs) are powerful robotic tools used by military and civilian communities alike. Advances in lightweight materials and commercial electronics have increased the range, payload, and reliability of the crafts. Popular applications for UAVs have been military reconnaissance, environmental surveying, aerial photography, and competitive racing.
%
% Much research has been conducted in the areas of UAV navigation, guidance, and control to promote the use of UAVs for more complex tasks. 
% 
% Turning high level tasks such as obstacle avoidance and path following into lower level guidance and control efforts has proven to be difficult and there appears to be no one-fits-all solution. Traditional control systems have been adapted for tasks such as following a moving path \cite{oliveira_moving_2016}, but prove to be fairly complicated and difficult to implement. 
% 
% Researches have found success in vector field methods for guidance and control.
% 
%  Vector fields provide continuous directional guidance to low-level control systems and have been effective at converging to and following a path of primitives \cite{nelson_cooperative_2005}, tracking uncertain moving targets \cite{chen_tracking_2009}, and for obstacle avoidance \cite{chen_uav_2013}. Several methods have been used in literature to construct vector fields consisting primarily of the Lyapunov method \cite{nelson_cooperative_2005}\cite{griffiths_vector_2006}\cite{frew_cooperative_2007}\cite{frew_lyapunov_nodate}  and the intersection of surfaces method \cite{goncalves_artificial_2009}. The vector field method allows for obstacle and goal fields to be generated separately and summed to yield a resulting guidance. Determining the weight of each field and the parameters that define circulation and convergence of the vector fields has yet to be investigated. Current field summation has been avoided due to the possibilities for singularities \cite{nelson_cooperative_2005}, which may be prevented with proper decay and vector field parameter tuning. The method presented in \cite{goncalves_artificial_2009} provides a convenient method for building vector fields consisting of convergence, circulation, and time varying terms. \textbf{The proposed research seeks to provide a goal and obstacle vector field summation strategy that improves circumnavigation around circular and elliptical obstacles}. 


% UAVs definition and uses
% How they do what they do (high level commands to fly-able path, guidance, control)
% Turning high lvel tasks such as obstacle avoidance and complex path following into guidance and lower level control has proven to be difficult. 
% Researches have found success in vector field methods for guidance and control.
%


Fixed wing Unmanned Aerial Vehicles (UAVs) have had an increase use in combat due to their outstanding endurance and autonomy. UAVs are well suited for surveillance and reconnaissance in hostile environments since no on-board pilot is required and missions can be programmed remotely. On-board autopilot systems consisting of path planners, navigation, and guidance are responsible for taking remotely programmed missions and producing flyable paths to a low level controller. Most tasks for UAVs can be framed as a path following problem which has been studied extensively in literature. Control systems have been developed for converging to and following moving paths \cite{oliveira_moving_2016}, but are not necessarily ideal for avoiding obstacles since the path must be recalculated. Much success has been found in vector field guidance methods for converging to and following paths, disturbance rejection, and obstacle avoidance. Vector fields provide continuous directional guidance to low-level control systems and have been effective at converging to and following a path of primitives \cite{nelson_cooperative_2005}, tracking uncertain moving targets \cite{chen_tracking_2009}, and for obstacle avoidance \cite{chen_uav_2013}. Several methods have been used in literature to construct vector fields consisting primarily of the Lyapunov method \cite{nelson_cooperative_2005}\cite{griffiths_vector_2006}\cite{frew_cooperative_2007}\cite{frew_lyapunov_nodate}  and the intersection of surfaces method \cite{goncalves_artificial_2009}. Intersection of surfaces method generates a field that converges to and follows a path by summing convergence, circulation, and time varying terms. The same method has been applied to generate a strictly repulsive vector field for obstacles by changing the sign of the convergence term [Wilhelm, Wambold, Clem]. Adding circulation to a vector field could reduce the distance to circumnavigate an obstacle in comparison to a strictly repulsive field. Vector field parameters have been kept constant and do not consider the state of the vehicle. The state of the UAV could be used to determine the strength of each vector field parameter and the decay function that represents the obstacle to reduce distance traveled. \textbf{The proposed research seeks to provide a goal and obstacle vector field parametrization and decay strategy that improves circumnavigation around circular and elliptical obstacles}.


\section{Methods Overview}
%Developing a goal and obstacle vector field summation strategy will be accomplished in three phases. First, singularities described in literature will be reproduced and show qualitatively how adding circulation to an obstacle field promotes circumnavigation. Next, the combination of decay functions and vector field parameters that improve obstacle avoidance will be investigated. Lastly, the modified vector field guidance will be tested on a mobile robot platform operating under the same constraints as a fixed wing will be used to validate the algorithm. 

Developing vector field parametrization and decay strategy for circumnavigating an obstacle will be accomplished in three phases. First, vector fields will be produced for circular and elliptical objects using the intersection of surfaces method. Obstacles will be given a strictly repulsive field and summed with a constant goal field resulting in several singularities. Next, a modified vector field will be investigated by varying the vector field parameters and decay as a function of a UAV state for the purposes of reducing the circumnavigation distance and eliminating singularities. A UAV represented by a Dubins airplane model will be given guidance by the field and the distance measured to determine which combinations of parameter and decay functions produce the least distance. Lastly, the algorithm will be implemented in a mobile robot to validate the modified guidance as a real-time path following and obstacle circumnavigating algorithm. 


\section{Phase I}
\textbf{The objective of Phase I is to recreate vector fields for circular and elliptical obstacles and demonstrate the possibility for singularities.} The intersection of two surfaces method for generating a path following vector field can also be used to generate repulsive vector fields by changing the sign of the convergence parameter. Two repulsive fields will be investigated consisting of circular and elliptical shapes. The equations for each field will be derived and it will be shown how modifying the convergence and circulation parameters effect the shape of a vector field. Lastly, the repulsive vector fields will be placed inside of goal vector field of constant heading to characterize the possibility to encounter singularities when summing opposing fields. 

\section{Phase II}
\textbf{The objective of Phase II is to investigate methods for selecting vector field parameters and decay functions to improve obstacle circumnavigation.} Obstacle vector fields placed in a goal field will theoretically put the UAV on a path to eventually circumnavigate an obstacle and reach the goal. The direction and intensity in which the UAV will maneuver around the obstacle is dependent on the weight of the convergence and circulation terms of a vector field. Calculating the convergence and circulation as a function of vehicle state may improve performance in comparison to a static hand-picked set of convergence and circulation terms. Common parameters used by autopilots such as heading, closing rate, position, and range will be used to develop functions that modify an obstacles vector field components. Decay functions for the obstacle and goal vector field weights will be investigated. Decay functions have been used for obstacles in [Wilhelm, Wambold, Clem] to repel a fixed wing UAV from an obstacle while circumnavigating a moving target, however weights for the goal field was not considered. Adding decay functions for both a goal and obstacle field prior to summing the two may help prevent singularities and deadzones. 


The modified field will be compared against a strictly repulsive field in simulation. Each simulation will consist of a circular or elliptical obstacle placed in a map where a constant goal vector field is present. A UAV modeled as a Dubins airplaine will use the resulting vector field guidance to navigate the map. The total distance traveled will be used to determine if a parametrized and decay function modified obstacle vector field improves the performance.


\section{Phase III}
\textbf{The objective of Phase III is to validate parametrized vector field circulation model through mobile robot experiments.} The modified obstacle vector field simulation scenarios from Phase II will be programmed into Python for easy implementation on a mobile robot. Simulations performed in Python will be compared against those done in MATLAB to ensure the algorithm is correct. The Python vector field guidance will be copied to a raspberry pi which receives position data from a Vicon motion capture system. Using the position and heading from the Vicon system, the modified guidance algorithm will calculate a commanded heading. The guidance command will be given to a low level controller that will calculate the necessary PWM commands to achieve the desired guidance.


\section{Summary of Objectives}
\textbf{Phase I:}
\begin{itemize}
\item The objective of Phase I is to recreate vector fields for circular and elliptical obstacles and demonstrate the possibility for singularities
\end{itemize}

\textbf{Phase II:}
\begin{itemize}
	\item The objective of Phase II is to investigate methods for selecting vector field parameters and decay functions to improve obstacle circumnavigation
\end{itemize}

\textbf{Phase III:}
\begin{itemize}
\item The objective of Phase III is to validate parametrized vector field circulation model through mobile robot experiments
\end{itemize}




\subsection{Deliverables}
The objectives listed above are put fourth so that the following deliverable can be presented at the conclusion of the research. \textbf{The proposed research seeks to provide a goal and obstacle vector field summation strategy that improves circumnavigation around circular and elliptical obstacles.}



\chapter{Literature Review}
\section{Introduction to Literature Review}

% The following sections provide a discussion on the literature in regards to the modeling of fixed wing UAVs and the navigation, guidance, and control systems that govern the behavior of the crafts
% 
%

The following sections provide a discussion on the literature in regards to the modeling of fixed wing UAVs and the navigation, guidance, and control systems that govern the behavior of the crafts. First, the fixed wing is introduced and a brief discussion on the modeling techniques commonly used for navigation, guidance, and control design. An overview of the structure of autopilots that execute the models is presented. Then, an overview of navigation, guidance, and control is given along with the cutting edge techniques for guidance. (Revisit this section when a rough draft is complete)


\section{Fixed Wing Unmanned Aerial Vehicle}
\subsection{Introduction to Fixed Wing UAV}
%  Fixed wing UAVs have found significant use in surveillance and reconnaissance in hostile environments since no on-board pilot is required.
%  UAVs can be sent missions and piloted remotely, keeping soldiers out of harms way
%  Operators can be relieved easily when becoming fatigued without landing the UAV, allowing it to fly for extended missions, accomplishing more tasks
%  Fixed wings come in many different sizes and configurations but can be categorized into large and hand lanched varieties
%  Large fixed wing UAVs require long stretches of runway to takeoff and land, are gas powered, and can carry large payloads consisting of sensor packages, high resolution cameras, and scientific instrumentation shown in Figure \ref{fig:globalhawk}.

Fixed wing UAVs have been used for surveillance and reconnaissance in hostile environments since no on-board pilot is required. Ground stations allow for remote data collection and piloting, keeping soldiers out of harms way. During long endurance missions, operators can easily be changed out when fatigued allowing the UAV to remain in service for many hours beyond what a single pilot could safely handle. Keeping the UAV in service for longer allows more tasks to be accomplished. Fixed wing UAVs come in many different sizes and configurations, but can be categorized into large and hand-launched varieties. Large fixed wing UAVs require long stretches of runway to takeoff and land, are gas powered, and can carry large payloads consisting of sensor packages, high resolution cameras, and scientific instrumentation shown in Figure \ref{fig:fixedwings}a. Hand-launched UAVs can be easily broken down and carried by a single person and are often battery powered. Easily deployed, the hand-launched UAVs take-off by being thrown in a similar way as a javelin shown in Figure \ref{fig:fixedwings}b. The smaller form factor makes hand-launched UAVs well suited for short range reconnaissance and for carrying lightweight payloads with low to medium resolution cameras.

%Fixed wing UAVs have found uses in surveillance, reconnaissance, environmental surveying, aerial photography, and competitive racing. The crafts are well suited to carry out tasks that require long endurance and large payloads.
%Fixed wing UAVs come in many different sizes and configurations, but can be categorized into large and hand-launched varieties. Large fixed wing UAVs require long stretches of runway to takeoff and land, are gas powered, and can carry large payloads consisting of sensor packages, high resolution cameras, and scientific instrumentation shown in Figure \ref{fig:globalhawk}. Hand-launched UAVs can be easily broken down and carried by a single person and are often battery powered. Easily deployed, the hand-launched UAVs take-off by being thrown in a similar way as a javelin shown in Figure \ref{fig:handlaunched}. The smaller form factor makes hand-launched UAVs well suited for short range reconnaissance and for carrying lightweight payloads with low to medium resolution cameras. \\

\begin{figure}[H]
	\begin{subfigmatrix}{2}% number of columns
		\centering
		\subfigure []{\includegraphics[width=7.7cm] {PaperFigures/globalhawk}}
		\subfigure []{\includegraphics[width=7cm] {PaperFigures/handlaunched}}
		\hspace*{0mm}
	\end{subfigmatrix}
	\caption{Global Hawk and Hand Launched Fixed Wing UAV}
	\label{fig:fixedwings}
\end{figure}





Both large and hand-launched UAVs operate by using path planning, navigation, guidance, and control algorithms depicted in Figure \ref{fig:ngcflow}. Most UAV tasks can be framed as path following problems such as waypoint navigation or loitering. The UAVs path planner is responsible for generating a flyable path from the current state to the desired state while avoiding known obstacles. Guidance systems are responsible for producing guidance commands to the low level controller that guide the UAV towards the desired path. Navigation systems are responsible for estimating the state of the UAV to provide feedback to both guidance and control loops. For more detail on navigation, guidance, and control systems, refer to section \ref{NGC}. Each of the navigation, guidance, and control sub systems are typically not evaluated independently and must be integrated into a complete system to evaluate the performance as a whole. The most common ways to measure the performance of a UAV system are tracking error with respect to the desired path and control effort. Next, common UAV modeling techniques will be discussed.

%Both large and hand-launched UAVs operate under the same basic principle consisting of navigation, guidance, and control depicted in Figure \ref{fig:ngcflow}. The desired path to be flown is passed to a guidance algorithm whose purpose is to produce guidance ocmmands consumable by the low level-control system that puts the UAV on the desired path. Navigation systems are responsible for estimating the state of the UAV to provide feedback to both guidance and control loops. For more detail on NGC systems, refer to section \ref{NGC}. Each of the NGC sub systems are typically not evaluated independently and must be integrated into a complete NGC system to evaluate th performance as a whole. The most common ways to measure the performance of a UAV system are tracking error with respect to the desired path and control effort.

\begin{figure}[]
	\centering
	\includegraphics[width=0.7\linewidth]{PaperFigures/ngcFlow}
	\caption{}
	\label{fig:ngcflow}
\end{figure}



\subsection{Modeling}
% Modeling of fixed wing UAVs is often simplified
% High fidelity models are complex because they are high-order and non-linear
% The 6 degree of freedom crafts only have 4 degrees of controllable freedom resulting in state coupling which also complicates the models
% High order non-linear models can be simplified when developing control systems that accurately capture the dynamics of the UAV
% Beyond the control aspect of UAVs, the high order non-linear models are not always necessary. It is common in literature to simplify complex non-linear dynamics of a fixed wing UAV by implementing a 2D Dubins vehicle model \cite{chen_tracking_2009} \cite{liang_tangent_2017} \cite{nelson_cooperative_2005} \cite{griffiths_vector_2006} \cite{jung_unmanned_2016}. When simplified kinematics are used when developing navigation and guidance systems, it is assumed that there already exists a low level control loop that maintains vehicle stability and performs according to the kinematic model in Equations \ref{dubinsx}-\ref{dubinsy}.
%

Accurate models of fixed wing UAVs tend to be high order, coupled, non-linear systems that can be difficult to solve for. The complexity is due to the fact that a fixed UAV has 6 degrees of freedom with only 4 degrees of control and has non-linear aerodynamic external forces. Models can be linearized and simplified when designing control systems, but still maintain high complexity and require significant effort to implement. Beyond the control aspect of UAVs, the high order non-linear models are not always necessary. It is common in literature to simplify complex non-linear dynamics of a fixed wing UAV by implementing a 2D Dubins vehicle model \cite{chen_tracking_2009} \cite{liang_tangent_2017} \cite{nelson_cooperative_2005} \cite{griffiths_vector_2006} \cite{jung_unmanned_2016}. When simplified kinematics are used when developing navigation and guidance systems, it is assumed that there already exists a low level control loop that maintains vehicle stability and performs according to the kinematic model in Equations \ref{dubinsx}-\ref{dubinsy}.

%The fixed wing UAV is a 6 degree of freedom (DOF) system with four degrees of control consisting of pitch, roll, yaw, and thrust. Change in the control variables is done by actuating the elevator (pitch), ailerons (roll), rudder (yaw), and propeller or jet (thrust). Due to the under actuation and complex non-linear dynamics of the UAV, high fidelity models tend to be high order and non-linear. High order non-linear models can be linearized and simplified to be used in low level control systems to maintain vehicle stability. Beyond the control aspect of UAVs, the high order non-linear models are not always necessary. It is common in literature to simplify complex non-linear dynamics of a fixed wing UAV by implementing a 2D Dubins vehicle model \cite{chen_tracking_2009} \cite{liang_tangent_2017} \cite{nelson_cooperative_2005} \cite{griffiths_vector_2006} \cite{jung_unmanned_2016}. When simplified kinematics are used when developing navigation and guidance systems, it is assumed that there already exists a low level control loop that maintains vehicle stability and performs according to the kinematic model in Equations \ref{dubinsx}-\ref{dubinsy}. 


The position of the vehicle at any point in 2D Cartesian space is defined as $(x,y)$ and the velocity of a vehicle in each axis is defined as $(\dot{x},\dot{y})$.

\begin{equation}
\label{dubinsx}
\dot{x} = V\cos(\theta)
\end{equation}
\begin{equation}
\label{dubinsy}
\dot{y} = V\sin(\theta)
\end{equation}


The heading of the vehicle $\theta$ is most frequently used as the input $\boldsymbol{u} = \theta$. To more closely represent the dynamics of a fixed wing, constraints are given to the velocity $V$ and turning rate $\dot{\theta}$. 



\begin{equation}\label{minVel}
V \geq V_{stall}
\end{equation}
\begin{equation}\label{headingRate}
-\dot{\theta}_{max} \leq \dot{\theta} \geq \dot{\theta}_{max}
\end{equation}

The devices that are assumed to be in control of the UAV when designing guidance and navigation systems are called autopilots or flight controllers, which will be discussed briefly in the next section. 


\subsection{Autopilot}
The autopilot is a software and hardware package that brings navigation, guidance, and control techniques to realization. A typical autopilot system is depicted in Figure \ref{fig:pixhawk}, consisting of a hard shell protecting sensors with numerous input and output connectors for power, communication, and servo actuation. The flight controller is responsible for several major tasks including maintaining vehicle stability, turning high level commands into low level control effort, and recording and transmitting data to ground stations. Autopilots accomplish these tasks by implementing several layers of algorithms consisting of navigation, guidance, and control which will be discussed next.

\begin{figure}[h]
	\centering
	\includegraphics[width=0.3\linewidth]{PaperFigures/pixhawk}
	\caption{Pixhawk Flight Controller}
	\label{fig:pixhawk}
\end{figure}

\section{Navigation, Guidance and Control} \label{NGC}
\subsection{Introduction to NGC}
Nearly all of the systems that go into an autopilot system for a fixed wing UAV can be put under the category of Navigation, Guidance, or Control (NGC). Traditionally, NGC could be thought of separate but equally important algorithm layers that aid an autopilot in accomplishing a given task. Navigation is the study of measuring and filtering the state of a vehicle. Guidance produces a commanded state based on high level requests from a path planner or user. Control maintains vehicle stability while attempting to achieve the state commanded by the high level guidance system. Lately, the lines between guidance and control have become less clear as the systems become more integrated with each other, resulting in more complex capabilities. Continuing with the convention in literature, navigation will be discussed as a separate unit while guidance and control will be discussed simultaneously. Guidance and control will be discussed as a single system and two commonly researched algorithms, potential and vector field methods, will be discussed.

\subsection{Navigation}

Navigation is the study of measuring the position and attitude of the UAV for the purposes of building guidance and providing feedback for the low level control system. Modern flight controllers contain a package of sensors called an Inertial Measurement Unit (IMU) which consists of a 3-axis accelerometer, gyroscope, compass, and barometer. The accelerometer and gyroscope are measured to determine the translational and angular acceleration of the craft. The compass measures the heading, and the barometer measures air pressure which is correlated to altitude. GPS is often used to determine the local position of the UAV. One of the major challenges in navigation is the noise and uncertainty in sensor measurements. Filters are used to fuse information and provide more accurate estimates of the vehicles position and rate of change.


\subsection{Guidance and Control}

%Need more citations on traditional guidance systems for path following to add to the discussion
Guidance and control have traditionally been separate systems, but as the need for UAVs to perform complex tasks increases, so does the complexity of the control systems. A common high level request provided to a UAV is to follow an arbitrary path which can be done with traditional controls \cite{zhao_curved_2017}. A control system for following a moving path was developed in \cite{oliveira_moving_2016}, which was successful at reducing the tracking error in simulation and real flight tests. To accomplish path following of a mobile path, the control law becomes complex and less intuitive. 

%Needs a better transition and explaination of the control system presented in oliveira 
%Another argument can be made for the distinction between the methods - PF works well for singular point seeking and VF is path seeking
Instead of relying on complex control laws, there has been much research on methods that combine guidance and control.  Two categories will be discussed consisting of potential field and vector field methods. In literature, the two methods have been seen used interchangeably. In the work presented, potential field is in reference to a gradient potential converging to a local minimum while vector field is in reference to a space of vectors whose integral lines converge and follow a path. It can be argued that vector fields are essentially a potential field, but for organizational purposes, they will be referred to as completely different methods. 

\subsection{Potential Field}
Potential field is a real-time robotic manipulator algorithm that distributes the task of goal seeking and obstacle avoidance among multiple layers of control \cite{khatib_real-time_1986}. The robot's workspace is represented as a gradient potential of attractive and repulsive artificial forces that drive the robot to a desired state. Goals are represented as an attractive force while obstacles provide a repulsive force. The potential field is constructed by modeling the robots motion in terms of Lagrangian mechanics shown in Equation \ref{LeeMotion}. The Lagrangian is defined as the difference in kinetic energy $T(x,\dot{x})$ and potential energy $U(x)$ in a system. The goal of the system imparts a potential $U_{xd}$ while obstacles impart a repulsive potential $U_o$. 

\begin{equation}\label{LeeMotion}
\frac{d}{dt} \left(\frac{\partial L}{\partial \dot{x}}\right) - \frac{\partial L}{\partial x} = F
\end{equation}

\begin{equation}\label{lagrange}
L(x,\dot{x}) = T(x,\dot{x}) - U(x)
\end{equation}

\begin{equation}\label{potential}
U_{art}(x) = U_{xd} + U_o(x)
\end{equation}

Potential field has been shown to be successful at driving a robot from an initial state to a goal state while avoiding obstacles by taking the path of least action formed by the gradient. One of the weaknesses of potential field identified early on is the methods susceptibility to local minima, preventing the robot from reaching the desired global minima, or goal state \cite{koren_potential_1991}. Local minima can be avoided in the potential field method if navigation functions are used \cite{goerzen_survey_2010}.\\

Potential field is useful for point-to-point guidance and control which is often the task of UAVs traveling to waypoints, however it is often desired for a UAV to converge to and follow a path. 

%%%%%%%%%% Additional points and clarification needed here before moving on
%
% Address navigation functions
% Re-read Goerzens review article and understand the scope of PF to a better degree
% Improve the transition to vector field
%		- Potential fields have proven to be useful for point-to-point navigation, however lack the framework for one of the common applications for fixed wing UAVs, path following
%
%


\subsection{Vector Field}
{Introduction to Vector Field}

Several methods have been developed to generate vector fields that converge to and follow paths. Histogram virtual force field (VFF) breaks the workspace into discrete cells that contain information in regards to the certainty of the presence of an obstacle \cite{borenstein_real-time_1990} \cite{borenstein_vector_1991}. Cells containing an obstacle apply an artificial guidance force away from the cell. Goals apply a global attractive force that drive the robot towards the desired state. The resultant guidance vector for the VFF method is the sum of the contributions of obstacle repulsive cells and the globally attractive goal. Lyapunov and navigation functions have been successfully used to provide guidance for converging to and following a path. Nelson et al. developed a method for generating a vector field that converges to and follows a path for line and circular primitives \cite{nelson_cooperative_2005}. Sinks, dead zones, and singularities were avoided when constructing more complex flight paths from the primitives by only allowing a single field to be active at any time. A vector field construction method for curved paths was presented in \cite{griffiths_vector_2006} by extending Nelson's method. Elliptical paths have been generated by from primitives by applying coordinate transformations to the field \cite{frew_lyapunov_nodate}.




Vector fields have many advantages to traditional guidance. UAVs often encounter disturbances in the form of wind which can be difficult to plan for. In the event a UAV encounters wind disturbances and is pushed off course, the field will drive the vehicle back on course \cite{de_marina_guidance_2017}. In addition to path following, vector field has been used to track a moving target in 3D \cite{miao_orthogonal_2016}. When the location of the target is unknown, loitering about an uncertain target can be achieved by building a circular vector field and applying a linear coordinate transformation to form an elliptical loiter based on the uncertainty in targets position estimate \cite{frew_cooperative_2007}. 


%
% 3D Vector field 
% \cite{meenakshisundaram_vector_2010}
%
%
%

- Following curved paths in a constant wind \cite{griffiths_vector_2006}


%%%%%% Points to make for vector field
%
% - Disturbance rejection (wind)
% - Well suited for unplanned obstacle avoidance
% - Many applications are best described as path following instead of singular point convergence like potential field
% - Standoff tracking of uncertain targets
%
%

Another method for constructing a vector field is by forming integral curves that converge at the intersection of surfaces \cite{goncalves_artificial_2009}. 

- construct an n-dimensional vector field by forming integral curves that converge at the intersection of surfaces
- Field is a result of the sum of 3 terms
- Convergence
- Circulation
- Time varying\\


% Note, modify the field equation to what jay has in his paper
\begin{equation}\label{gonFieldeq}
\boldsymbol{u} = G\nabla_qV+H\wedge_{i=1}\nabla_q\alpha_i - M(\alpha)^{-1}\boldsymbol{a}(\alpha)
\end{equation}

- Define all variables
- u is the resulting 2x1 vector
- Gradient
- $alpha_i$ is a surface function
- Wedge product simplifies to cross product in 2d
- G, H, L scalar weighting quantities 
- The intersection of the surfaces represents a path to converge and follow
- Cylinder and plane example


\begin{figure}[h]
	\centering
	\includegraphics[width=0.7\linewidth]{PaperFigures/convergence}
	\caption{Convergence}
	\label{fig:convergence}
	
\end{figure}

\begin{figure}[h]
	\centering
	\includegraphics[width=0.7\linewidth]{PaperFigures/circulation}
	\caption{Circulation}
	\label{fig:circulation}
\end{figure}

\begin{figure}[h]
	\centering
	\includegraphics[width=0.7\linewidth]{"PaperFigures/time varying"}
	\caption{Time Varying}
	\label{fig:time-varying}
\end{figure}


\begin{figure}[h]
	\centering
	\includegraphics[width=0.7\linewidth]{PaperFigures/total}
	\caption{Total Field}
	\label{fig:total}
\end{figure}

\begin{itemize}
	\item Cooperative Standoff Tracking of Uncertain moving targets (2007, Frew)
	\item VF usefulness extended to loitering about an uncertain target
	\item Lyapunov vector field generation for a circular loiter
	\item Linear transformation applied to stretch the field into an ellipse shape
\end{itemize}





\section{Literature Review Summary}

\bibliographystyle{aiaa}   
\bibliography{bib}


\end{document}
